\documentclass[11pt]{article}

\usepackage{amsmath, amssymb}
\usepackage{graphicx}
\usepackage{geometry}
\usepackage{hyperref}
\usepackage{bm}
\usepackage{authblk}

\geometry{margin=1in}

\title{\textbf{Prime--Composite Early Symmetry (PCES) Theory:\\
A Number-Theoretic Analogue of Early-Universe Symmetry Breaking}}

\author[1]{Tareq Majeed Al-Karimi}
\author[2]{ChatGPT (AI Collaborator)}
\affil[1]{Thi Qar Oil Company, Mathematical Physics Research Division}
\affil[2]{OpenAI Research Assistance}

\date{\today}

\begin{document}

\maketitle

\begin{abstract}
We introduce the Prime--Composite Early Symmetry (PCES) Theory,
motivated by a rare natural-number symmetry between cumulative sums of
primes and composites that exists only for early $k$ and disappears permanently afterward.
This structure mirrors early-universe symmetry before inflation.  
We define a new observable, PCME (Prime--Composite Matching Energy), and show through
simulation that PCME is more sensitive than the classical T-Index.
\end{abstract}

\section{Introduction}

Recent discoveries in number theory revealed a rare equality between early
cumulative prime and composite sums...
[تكملة النص كما تريد]

\end{document}
